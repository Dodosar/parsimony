\documentclass[11pt]{article}

\usepackage{graphicx}
\usepackage{courier}

\title{TMD ReadMe}
\author{Adam Yedidia}

\begin{document}
    
\maketitle

This document is intended for users who want to know how to run or compile a TMD directory, without necessarily knowing how to program one themselves. To learn how to create a valid TMD directory, read \texttt{tmd\_doc.pdf}.

The top-level representation of TMD is a program written in the TMD language, which is a language designed to give the user a simple interface with which to program a multi-tape, 3-symbol Turing Machine with a function stack. Each of the many tapes in the execution of the program is infinite in one direction. \\

TMD code can be processed in two ways. First, it can be \emph{interpreted}; that is, it can be directly evaluated line-by-line. This is generally done to verify a program's correct behavior, and to correct errors which would result in thrown exceptions in the interpreter but might lead to undefined behavior in the compiled Turing machine (because the compiled Turing machine is optimized for parsimony, whereas the interpreter need not be). \\

Second, TMD code can be \emph{compiled} down to a description of a single-tape, 2-symbol Turing machine. It is highly recommended, however, to first interpret any piece of TMD code before compiling it, because the interpreter is much better for catching programming errors. The compiler is general-purpose and not restricted to compiling the programs discussed in this thesis. It is optimized to minimize the number of states in the resulting Turing machine, not to make the resulting Turing machine time- or space-efficient. \\

\section{Preparation}

Before a TMD directory can be processed in any way, first it must exist in a place where it can be found. Make sure that the target directory is located in the \texttt{parsimony/src/tmd/tmd\_dirs/} directory. (If you created your TMD directory via compilation from Laconic, it will already exist in the right place.)

\section{Interpretation}

To interpret a TMD 

\end{document}