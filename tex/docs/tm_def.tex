\documentclass[11pt]{article}

\usepackage{graphicx}
\usepackage{courier}
\usepackage{underscore}

\title{Turing Machine Definitions}
\author{Adam Yedidia}

\begin{document}
    
\maketitle

This document explains the formal definitions of Turing Machines as generated by this project. Note that most of this document also appears in \emph{A Relatively Small Turing Machine Whose Behavior Is Independent of Set Theory}, which can be found at: \\ \\
\texttt{parsimony/tex/busybeaver/busybeaver.pdf}

\section{2-Symbol Turing Machines}

This section explores how the 2-symbol Turing machines that are generated by this project are defined.

There are many slightly different definitions of Turing machines. \ For example, some definitions allow the machine to have multiple tapes; others only allow it to have one; some allow an arbitrarily large alphabet, while others allow only two symbols, and so on. \ In most research regarding Turing machines, mathematicians don't concern themselves with which of these models to use, because any one can simulate the others (usually efficiently). \ However, because this work is concerned with upper-bounding the exact number of states required to perform certain tasks, it's important to define the model precisely. \ The model we choose here is traditional for the Busy Beaver function.

Formally, a $k$-state Turing machine is a 7-tuple $M = (Q, \Gamma, b, \Sigma, \delta, q_0, F)$, where: \\ \\
$Q$ is the set of $k$ \emph{states} $\{q_0, q_1, \dots, q_{k-2}, q_{k-1}\}$ \\
$\Gamma = \{a, b\}$ is the set of \emph{tape alphabet symbols} \\
\texttt{a} is the \emph{blank symbol} \\
$\Sigma = \empty$ is the set of \emph{input symbols} \\\
$\delta = Q \times \Gamma \rightarrow (Q \cup F) \times \Gamma \times \{L, R\}$ is the \emph{transition function} \\
$q_0$ is the \emph{start state} \\
$F = \{\textrm{HALT}, \textrm{ERROR}\}$ is the set of \emph{halting transitions}. \\

A Turing machine's \emph{states} make up the Turing machine's easily-accessible, finite memory. \ The Turing machine's state is initialized to $q_0$.

The \emph{tape alphabet symbols} correspond to the symbols that can be written on the Turing machine's infinite tape.

In this work, all Turing machines are run on the all-\texttt{a} input.

The \emph{transition function} encodes the Turing machine's behavior. \ It takes two inputs: the current state of the Turing machine (an element of $Q$) and the symbol read off the tape (an element of $\Gamma$). \ It outputs three instructions: what state to enter (an element of $Q$), what symbol to write onto the tape (an element of $\Gamma$) and what direction to move the head in (an element of $\{L, R\}$). \ A transition function specifies the entire behavior of the Turing machine in all cases.

The \emph{start state} is the state that the Turing machine is in at initialization.

A \emph{halting transition} is a transition that causes the Turing machine to halt. 

\section{4-Symbol Turing Machines}

This section explores how the 4-symbol Turing machines that are generated by this project are defined.

a $k$-state, 4-symbol Turing machine is a 7-tuple $M = (Q, \Gamma, b, \Sigma, \delta, q_0, F)$, where: \\ \\
$Q$ is the set of $k$ \emph{states} $\{q_0, q_1, \dots, q_{k-2}, q_{k-1}\}$ \\
$\Gamma = \{\texttt{_}, \texttt{1}, \texttt{H}, \texttt{E}\}$ is the set of \emph{tape alphabet symbols} \\
\texttt{_} is the \emph{blank symbol} \\
$\Sigma = \empty$ is the set of \emph{input symbols} \\\
$\delta = Q \times \Gamma \rightarrow (Q \cup F) \times \Gamma \times \{L, -, R\}$ is the \emph{transition function} \\
$q_0$ is the \emph{start state} \\
$F = \{\textrm{HALT}, \textrm{ERROR}\}$ is the set of \emph{halting transitions}. \\

\end{document}